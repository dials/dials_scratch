\documentclass[a4paper, 11pt]{article}

\usepackage{amsmath}
\usepackage{amssymb}
\usepackage{graphicx}
\usepackage{fullpage}
\usepackage{parskip}
\usepackage[numbers]{natbib}
\usepackage{url}

\title{Conventions for DIALS developments}
\author{DIALS Developers}

% use bold face for vectors
\renewcommand{\vec}[1]{\mathbf{#1}}

% partial derivatives
\newcommand{\pder}[2][]{\frac{\partial#1}{\partial#2}}

\begin{document}
\maketitle
\section{Coding standards}

FIXME something about standards for coding in here.

\section{Coordinate frames}

\subsection{The diffractometer equation}

We use the vector $\vec{h}$ to describe a position in \emph{fractional reciprocal space}
in terms of the reciprocal lattice basis vectors $\vec{a^*}$, $\vec{b^*}$ and
$\vec{c^*}$.

\begin{equation}
\label{eq:miller_index}
\vec{h} = \begin{pmatrix}
h \\
k \\
l \\
\end{pmatrix} = h \vec{a^*} + k \vec{b^*} + l \vec{c^*}
\end{equation}

The special positions at which $h$, $k$ and $l$ are integer define the 
\emph{reciprocal lattice points} for which $(hkl)$ are the \emph{Miller
indices}.

The basic diffractometer equation relates a position $\vec{h}$ to a position
$\vec{r_\phi}$ in \emph{Cartesian reciprocal space}. This space is defined so
that its axes coincide with the axes of the \emph{laboratory frame}. The
distinction is necessary because distances in reciprocal space are measured in
units of {\AA}$^{-1}$. However, for convenience it is often acceptable to refer
to either Cartesian reciprocal space or the real space laboratory frame as the
``lab frame'', when the correct choice is clear by context. The diffractometer
equation is

\begin{equation}
\label{eq:diffractometer}
\vec{r_\phi} = \mathbf{R} \mathbf{A} \vec{h}
\end{equation}

where $\mathbf{R}$ is the \emph{goniostat rotation matrix} and $\mathbf{A}$ is
the \emph{crystal setting matrix}, while its inverse $\mathbf{A}^{-1}$ is referred
to as the \emph{indexing matrix}. The product $\mathbf{A} \vec{h}$ may be written
as $\vec{r_0}$, which is a position in the \emph{$\phi$-axis frame}, a
Cartesian frame that coincides with the laboratory frame at a rotation angle of
$\phi=0$. This makes
clear that the setting matrix does not change during the course of a rotation
experiment (notwithstanding small ``misset'' rotations --- see
Section~\ref{sec:U_matrix}).

For an
experiment performed using the rotation method we use here $\phi$ to refer
to the angle about the actual axis of rotation, even when this is effected
by a differently labelled axis on the sample positioning equipment (such as an
$\omega$ axis of a multi-axis goniometer). Only in code specifically dealing with
sample positioning equipment might we need to redefine the labels of axes.
Outside of such modules, the rotation angle is $\phi$ and the axis of rotation is
$\vec{e}$, which together with the definition of the laboratory frame determine
the rotation matrix $\mathbf{R}$.

\subsection{Orthogonalisation convention}

Following \citet{Busing1967diffractometers} we may decompose the setting matrix
$\mathbf{A}$ into the product of two matrices, conventionally labelled $\mathbf{U}$
and $\mathbf{B}$. We name $\mathbf{U}$ the \emph{orientation matrix} and
$\mathbf{B}$ the \emph{reciprocal space orthogonalisation matrix}. These names
are in common, but not universal use. In particular, some texts (for example
\citealt{Paciorek1999geometry}) refer to the product (i.e. our setting matrix)
as the ``orientation matrix''.

Of these two matrices, $\mathbf{U}$ is a
pure rotation matrix and is dependent on the definition of the lab frame, whilst
$\mathbf{B}$ is not dependent on this definition. $\mathbf{B}$ does depend however
on a choice of orthogonalisation convention, which relates $\vec{h}$ to a position
in the \emph{crystal-fixed Cartesian system}. The basis vectors of this orthogonal
Cartesian frame are fixed to the reciprocal lattice \textit{via} this convention.

There are infinitely many ways that $\mathbf{A}$ may be decomposed into a pair
$\mathbf{U} \mathbf{B}$. The symbolic expression of $\mathbf{B}$ is simplest when
the crystal-fixed Cartesian system is chosen to be aligned with crystal real or
reciprocal space axes. For example, \citet{Busing1967diffractometers} use a frame
in which the basis vector $\vec{i}$ is parallel to reciprocal lattice vector
$\vec{a^*}$, while $\vec{j}$ is chosen to lie in the plane of $\vec{a^*}$ and
$\vec{b^*}$. Unfortunately, this convention is then disconnected from the
standard \emph{real space} orthogonalisation convention, usually called the \emph{PDB
convention} \citep{PDB1992atomic}. This standard is essentially universal in
crystallographic software for the transformation of fractional crystallographic
coordinates to positions in orthogonal space, with units of \AA. In particular, it
is the convention used in the cctbx \citep{GrosseKunstleve2002cctbx}. The
convention states that the orthogonal coordinate $x$ is determined from a
fractional coordinate $u$ by

\begin{equation}
\label{eq:realspaceortho}
\vec{x} = \mathbf{O} \vec{u}
\end{equation}

where the matrix $O$ is the \emph{real space orthogonalisation matrix}. This matrix
transforms to a crystal-fixed Cartesian frame that is defined such that its basis
vector $\vec{i}$ is parallel to the real space lattice vector $\vec{a}$, while
$\vec{j}$ lies in the $(\vec{a}, \vec{b})$ plane. The elements of this matrix
made explicit in a compact form are

\begin{equation}
\label{eq:realspaceorthomatrix}
\mathbf{O} = 
\begin{pmatrix}
a & b\cos{\gamma} &  c\cos{\beta} \\
0 & b\sin{\gamma} & -c\sin{\beta}\cos{\alpha^*} \\
0 & 0             &  c\sin{\beta}\sin{\alpha^*} \\
\end{pmatrix}
\end{equation}

It is desirable to specify our \emph{reciprocal space} orthogonalisation convention in 
terms of this real space orthogonalisation convention. \citet{giacovazzo2002fundamentals}
derives relationships between real and reciprocal space. Of particular interest
from that text we have

\begin{eqnarray}
\label{eq:realreciprocaltransforms}
\vec{x} & = & \mathbf{M}^\mathsf{T} \vec{x}^\prime \nonumber \\
\vec{x^*} & = & \mathbf{M}^{-1} \vec{x^*}^\prime
\end{eqnarray}

By analogy, equate $\vec{x^*}^\prime$ with $\vec{h}$ and $\mathbf{B}$ with
$\mathbf{M}^{-1}$. Also equate $\mathbf{M}^\mathsf{T}$ with $\mathbf{O}$ and 
$\vec{x}^\prime$ with $\vec{u}$. We then see that

\begin{equation}
\label{eq:reciprocalortho}
\mathbf{B} = \left( \mathbf{O}^{-1} \right)^\mathsf{T} = \mathbf{F}^\mathsf{T}
\end{equation}

where $\mathbf{F}$ is designated the \emph{real space fractionalisation matrix}.
This is easily obtained in cctbx by a method of a \verb!cctbx.uctbx.unit_cell!
object.

A symbolic expansion of $\mathbf{B}$ in terms of the real space unit cell
parameters will be required for the calculation of the derivatives of
$\mathbf{B}$ wrt these parameters. An expression for $\mathbf{F}$ is given by
\citet{ruppwebb_transformation} from which we derive $\mathbf{B}$ simply:

\begin{equation}
\label{eq:recipspaceorthomatrix}
\mathbf{B} = 
\begin{pmatrix}
\frac{1}{a} &
0 &
0 \\
-\frac{\cos{\gamma}}{a\sin{\gamma}} &
\frac{1}{b\sin{\gamma}} &
0 \\
\frac{bc}{V}\left( \frac{\cos{\gamma} \left( \cos{\alpha} - \cos{\beta}\cos{\gamma} \right)}{\sin{\gamma}} - \cos{\beta}\sin{\gamma} \right) &
-\frac{ac \left( \cos{\alpha} - \cos{\beta}\cos{\gamma} \right)}{V\sin{\gamma}} &
\frac{ab\sin{\gamma}}{V} \\
\end{pmatrix}
\end{equation}

with $V = abc \sqrt{ 1 - \cos^2{\alpha} - \cos^2{\beta} - \cos^2{\gamma} + 2 \cos{\alpha}\cos{\beta}\cos{\gamma}}$

TODO This expression should be tested!

\subsection{Orientation matrix} \label{sec:U_matrix}

The matrix $\mathbf{U}$ `corrects' for the orthogonalisation convention
implicit in the choice of $\mathbf{B}$. As the crystal-fixed Cartesian system and
the $\phi$-axis frame are both orthonormal, Cartesian frames with the same scale,
it is clear that $\mathbf{U}$ must be a pure rotation matrix. Its elements are
clearly dependent on the mutual orientation of these frames.

It is usual to think of the orientation as a fixed property of the ``sequence''.
In practice the orientation is parameterised such that it becomes a function
of time, to account for crystal slippage (the true degree of this is unknown but
expected to be small; Mosflm uses crystal orientation parameters to account for
inadequacies in other aspects of the experimental description). To reconcile
these points, the current orientation may be expanded into a fixed, datum part
and a variable time-dependent part that is parameterised. That gives

\begin{equation}
\vec{r_\phi} = \mathbf{\Psi}\mathbf{R}\mathbf{U_0}\mathbf{B}\vec{h}
\end{equation}

where $\Psi$ is the combined rotation matrix for the misset expressed as
three angles, $\psi_x$, $\psi_y$ and $\psi_z$ in the laboratory frame.

In Mosflm these angles are converted to their equivalents in the $\phi-$axis 
frame, where:

\begin{equation}
\vec{r_\phi} = \mathbf{R}\mathbf{\Phi}\mathbf{U_0}\mathbf{B}\vec{h}
\end{equation}

At this stage it is unclear which set of angles are the best choice for
parameterisation of the crystal orientation.

\subsection{The laboratory frame}

An important design goal of the DIALS project is that all algorithms should be
fully vectorial. By this we mean that it should be possible to change the
reference frame arbitrarily and all calculations should work appropriately in
the new frame.

FIXME Note this is not currently true in the case of translations. We assume that
the intersection of the crystal and beam occurs at the origin of our laboratory
system. Is this going to be a problem?

Nevertheless, it is useful to adopt a particular standard frame of reference for
meaningful comparison of results, communication between components of the
software and for an agreed definition of what the laboratory consists of
(incompatible definitions can be reasonably argued for, such as that it should
be either fixed to the detector, or to the rotation axis and beam).

In the interests of standardisation, we choose to adopt the Image CIF (imgCIF)
reference frame \citep{Berstein2006imagedata, Hammersley2006imgCIF}.

FIXME Some expansion of that here.

\subsection{Summary of coordinate frames}
\begin{itemize}
 \item $\vec{h}$ gives a position in \emph{fractional reciprocal space}, fixed 
       to the crystal.
 \item $\mathbf{B}\vec{h}$ gives that position in the \emph{crystal-fixed Cartesian system}
       (basis aligned to crystal axes by the orthogonalization convention)
 \item $\mathbf{UB}\vec{h}$ gives the \emph{$\phi$-axis frame} (rotates with the crystal, axes
       aligned to lab frame at $\phi=0$)
 \item $\mathbf{RUB}\vec{h}$ gives \emph{Cartesian reciprocal space} (fixed wrt the laboratory)
 \item Diffraction geometry (Section~\ref{sec:diff_geom}) relates this to the direction of
       the scattering vector $\vec{s}$ in the \emph{laboratory frame}
 \item Projection along $\vec{s}$ impacts an \emph{abstract sensor frame}
       giving a 2D position of the reflection position on a sensor.
 \item This position is converted to the \emph{pixel position} for the 2D position
       on an image in number of pixels (starts 0,0 at origin?)
\end{itemize}

\section{Diffraction geometry} \label{sec:diff_geom}

This is described in detail in the document ``Reflection prediction''. Perhaps
the introductory parts of that document should be moved here.

\section{Detector model}

A composite detector may be composed of multiple `sensors'. Each sensor is supposed to be a single
detective surface that may be reasonably described mathematically by a plane and limits in orthogonal
directions thus forming a letterbox in space. This corresponds closely to the idea of an `abstract
detector' 

It is expected that any arrangement of multiple sensors with rigidly fixed offsets and mutual
orientations will be described by a single set of parameters and constitutes a single detector parameterisation.

There is a distinction between a `detector model' and a `detector parameterisation'
that risks some confusion. Here we attempt some strict definitions to clarify 
that distinction.

A detector model implements the interface by which all detector-related operations are
performed, and all detector-related data is accessed. A detector model will contain a
list of sensors (one or many) which constitute the detective surfaces of that detector.
At this stage we assume \emph{only one detector exists} in the experiment, thus there
is only one detector model. For the purposes of reflection prediction and refinement,
this singleton detector model is a directory of all of the sensors in the experiment.
This means that coordinates given as $X$, $Y$, $i$, where $i$ is a unique sensor number
fully cover all of the detector space.

A `detector parameterisation' also contains a grouping of sensors. This group is described by a single
set of parameters and is therefore supposed to be a single physical entity, which is internally
static, but may be oriented arbitrarily within the laboratory frame. However, it is possible
that this detector parameterisation does not cover the full detector model. There may be
multiple detector parameterisations, in the case where parts of a detector may move relative
to one another (refer to the CXI PAD, although even in such cases it is likely to be a very unusual situation that warrants such
movements as degrees of freedom).

We also state that a sensor may only belong to one
parameterisation at any one time (overlapping groups of sensors are not allowed), or none
(in the case where we are absolutely certain of the sensor location and orientation and this
need not be refined).

This should cover almost all use cases.

\bibliographystyle{plainnat}
\bibliography{../dials}

\end{document}
