%------------------------------------------------------------------------------
% Template file for the submission of papers to IUCr journals in LaTeX2e
% using the iucr document class
% Copyright 1999-2013 International Union of Crystallography
% Version 1.6 (28 March 2013)
%------------------------------------------------------------------------------

\documentclass[preprint]{iucr}              % DO NOT DELETE THIS LINE

  %----------------------------------------------------------------------------
  % Extra packages
  %----------------------------------------------------------------------------
  \usepackage{graphicx}         % For graphics
  \usepackage{mathtools}        % Math stuff
  \usepackage{bm}               % Bold in maths
  \usepackage{listings}         % Code snippets
  \usepackage{bold-extra}       % Bold mono space for code snippets
  \usepackage{url}              % For URLs
  \usepackage{xspace}           % Spacing in macros
  \usepackage{color}            % Colours
  \usepackage{textcomp}         % Required by listings when upquote=true
  \usepackage{multirow}         % Multirow spanning cells in tables
  \usepackage{siunitx}          % Proper formatting for units

     %-------------------------------------------------------------------------
     % Information about journal to which submitted
     %-------------------------------------------------------------------------
     \journalcode{J}              % Indicate the journal to which submitted
                                  %   A - Acta Crystallographica Section A
                                  %   B - Acta Crystallographica Section B
                                  %   C - Acta Crystallographica Section C
                                  %   D - Acta Crystallographica Section D
                                  %   E - Acta Crystallographica Section E
                                  %   F - Acta Crystallographica Section F
                                  %   J - Journal of Applied Crystallography
                                  %   M - IUCrJ
                                  %   S - Journal of Synchrotron Radiation

  %----------------------------------------------------------------------------
  % Bits of formatting used throughout document
  %----------------------------------------------------------------------------
  % program names
  \newcommand{\cctbx}{\emph{cctbx}\xspace}
  \newcommand{\rstbx}{\emph{rstbx}\xspace}
  \newcommand{\dxtbx}{\emph{dxtbx}\xspace}
  \newcommand{\lstbx}{\emph{lstbx}\xspace}
  \newcommand{\dials}{\emph{DIALS}\xspace}
  \newcommand{\dialsimport}{\emph{dials.import}\xspace}
  \newcommand{\dialsindex}{\emph{dials.index}\xspace}
  \newcommand{\dialsrefine}{\emph{dials.refine}\xspace}
  \newcommand{\dialsfindspots}{\emph{dials.find\_spots}\xspace}
  \newcommand{\xiatwo}{\emph{xia2}\xspace}
  \newcommand{\xds}{\emph{XDS}\xspace}
  \newcommand{\mosflm}{\emph{MOSFLM}\xspace}
  \newcommand{\madnes}{\emph{MADNES}\xspace}
  \newcommand{\fastmcd}{\emph{FAST-MCD}\xspace}
  \newcommand{\aimless}{\emph{AIMLESS}\xspace}
  % derivatives
  \newcommand{\pder}[2][]{\frac{\partial#1}{\partial#2}}
  \newcommand{\tder}[2][]{\frac{\mathrm{d}#1}{\mathrm{d}#2}}
  % use bold face for vectors
  \renewcommand{\vec}[1]{\mathbf{#1}}
  \newcommand{\uvec}[1]{\hat{\mathbf{#1}}}
  \newcommand{\mat}[1]{\mathbf{#1}}
  \newcommand{\rmat}[1]{\mat{R}_{#1}}

  % use to fix order in bibtex entries
  \newcommand{\mockalph}[1]{}

  % FIXMEs and visible comments
  \newcommand\fixme[1]{\textcolor{red}{#1}}

\begin{document}                  % DO NOT DELETE THIS LINE

     %-------------------------------------------------------------------------
     % The introductory (header) part of the paper
     %-------------------------------------------------------------------------

     % The title of the paper. Use \shorttitle to indicate an abbreviated title
     % for use in running heads (you will need to uncomment it).

\title{Detector metrology with DIALS}
%\shorttitle{Short Title}

     % Authors' names and addresses. Use \cauthor for the main (contact) author.
     % Use \author for all other authors. Use \aff for authors' affiliations.
     % Use lower-case letters in square brackets to link authors to their
     % affiliations; if there is only one affiliation address, remove the [a].

  \cauthor[a,b]{David G.}{Waterman}{david.waterman@stfc.ac.uk}{}
  \author[c]{Graeme}{Winter}
  \author[c]{Richard J.}{Gildea}
  \author[c,d]{James M.}{Parkhurst}
  \author[e]{Aaron S.}{Brewster}
  \author[e]{Nicholas K.}{Sauter}
  \cauthor[c]{Gwyndaf}{Evans}{gwyndaf.evans@diamond.ac.uk}{}

  \aff[a]{STFC Rutherford Appleton Laboratory,
    Didcot,
    OX11 0QX,
    \country{UK}}

  \aff[b]{CCP4,
    Research Complex at Harwell,
    Rutherford Appleton Laboratory,
    Didcot,
    OX11 0FA,
    \country{UK}}

  \aff[c]{Diamond Light Source Ltd,
    Harwell Science and Innovation Campus,
    Didcot,
    OX11 0DE,
    \country{UK}}

  \aff[d]{MRC Laboratory of Molecular Biology,
    Francis Crick Avenue,
    Cambridge,
    CB2 0QH,
    \country{UK}}

  \aff[e]{Lawrence Berkeley National Laboratory,
    Berkeley, California 94720, \country{USA}}

  \shortauthor{Waterman \emph{et al.}}

     % Use \shortauthor to indicate an abbreviated author list for use in
     % running heads (you will need to uncomment it).

%\shortauthor{Soape, Author and Doe}

     % Use \vita if required to give biographical details (for authors of
     % invited review papers only). Uncomment it.

%\vita{Author's biography}

     % Keywords (required for Journal of Synchrotron Radiation only)
     % Use the \keyword macro for each word or phrase, e.g.
     % \keyword{X-ray diffraction}\keyword{muscle}

%\keyword{keyword}

     % PDB and NDB reference codes for structures referenced in the article and
     % deposited with the Protein Data Bank and Nucleic Acids Database (Acta
     % Crystallographica Section D). Repeat for each separate structure e.g
     % \PDBref[dethiobiotin synthetase]{1byi} \NDBref[d(G$_4$CGC$_4$)]{ad0002}

%\PDBref[optional name]{refcode}
%\NDBref[optional name]{refcode}

\maketitle                        % DO NOT DELETE THIS LINE

\begin{synopsis}
Supply a synopsis of the paper for inclusion in the Table of Contents.
\end{synopsis}

\begin{abstract}
Abstract goes here.
\end{abstract}


     %-------------------------------------------------------------------------
     % The main body of the paper
     %-------------------------------------------------------------------------
     % Now enter the text of the document in multiple \section's, \subsection's
     % and \subsubsection's as required.

\section{Introduction}

\fixme{author list just copied from \dialsrefine paper so far}

\fixme{ref this https://www.osapublishing.org/oe/abstract.cfm?uri=oe-23-22-28459}

Start by brief discussion of area detectors in crystallography:

\begin{itemize}
  \item Bigger is better
  \item For fast detectors (i.e. faster than image plates) this has been
  achieved by constructing detectors as modular arrays of units
  \item For CCD detectors significant distortions caused by coupling to
  phosphor required correction in software. The modular structure is no
  longer obvious in the stitched-together corrected images
  \item For PADs (Si bump-bonded onto large CMOS chips) the array of pixels
  within a panel has a regular size (to what tolerance?) so that the
  position of photons impinging upon the front surface of the panel can be
  inferred from the counts present in the image file, once the parallax
  correction effect is taken into account.
  \item The arrangement of whole modules has tolerances on a different scale
  to the bump bonding process - macroscopic engineering problem rather than
  microscopic (how can this be expressed nicely?)
  \item Shifts of significant fractions of a pixel may be present between a
  panel's real and its canonical position.
  \item Discuss how these whole module offsets can be measured at the
  factory (this is the true meaning of 'metrology') and correction tables
  may be supplied by the detector manufacturer
  \item But here we discuss an alternative procedure in which the
  appropriate corrections can be discovered by an analysis of diffraction
  data as a part of commissioning.
\end{itemize}

Refinement is good \cite{Waterman2016}.

First Pilatus references \cite{Hulsen2006Pilatus1M,Broennimann2006}

P1M distortion correction \cite{Hulsen2005P1MDistortion}

``Within a module of the detector, the pixel positions are accurate to
sub-micrometre precision because the sensor is fabricated with
photolithographic processes'' \cite{Hulsen2006Pilatus1M}

``modules are mounted to a high-precision mechanical frame to realize
multimodules setups'' (Experience and Results from the 6 Megapixel PILATUS
System. Bergamaschi 2007)

``high-precision mechanical frame.'' \cite{kraft2009calibration}

From the Pilatus user manual (\url{http://www.esrf.eu/files/live/sites/www/files/UsersAndScience/Experiments/MX/New_Web_Files/Beamlines/Hardware_components/Detectors/User_Manual-PILATUS2-V1_4.pdf}):
3) Distortion (only for Multi Module Systems)
A text file with a map of the offset in position and angle of each module with
respect to some common origin will be provided.
The processing program must incorporate this information, using it as a
lookup table to map sought reciprocal space positions onto detector positions.
However, due to the high manufacturing quality of the detector and the direct
detection, geometrical distortions are minimal and excellent results are
obtained without correction.


Text text text text text text text text text text text text text text
text text text text text text text.

\subsection{Title}

Text text text text text text text text text text text text text text
text text text text text text text.

\subsubsection{Title}

Text text text text text text text text text text text text text text
text text text text text text text.


     % Appendices appear after the main body of the text. They are prefixed by
     % a single \appendix declaration, and are then structured just like the
     % body text.

\appendix
\section{Appendix title}

Text text text text text text text text text text text text text text
text text text text text text text.

\subsection{Title}

Text text text text text text text text text text text text text text
text text text text text text text.

\subsubsection{Title}

Text text text text text text text text text text text text text text
text text text text text text text.


     %-------------------------------------------------------------------------
     % The back matter of the paper - acknowledgements and references
     %-------------------------------------------------------------------------

     % Acknowledgements come after the appendices

\ack{Acknowledgements}

     % References are at the end of the document, between \begin{references}
     % and \end{references} tags. Each reference is in a \reference entry.

\referencelist[metro0]

%\begin{references}
%\reference{Author, A. \& Author, B. (1984). \emph{Journal} \textbf{Vol},
%first page--last page.}
%\end{references}

     %-------------------------------------------------------------------------
     % TABLES AND FIGURES SHOULD BE INSERTED AFTER THE MAIN BODY OF THE TEXT
     %-------------------------------------------------------------------------

     % Simple tables should use the tabular environment according to this
     % model

\begin{table}
\caption{Caption to table}
\begin{tabular}{llcr}      % Alignment for each cell: l=left, c=center, r=right
 HEADING    & FOR        & EACH       & COLUMN     \\
\hline
 entry      & entry      & entry      & entry      \\
 entry      & entry      & entry      & entry      \\
 entry      & entry      & entry      & entry      \\
\end{tabular}
\end{table}

     % Postscript figures can be included with multiple figure blocks

\begin{figure}
\caption{Caption describing figure.}
\includegraphics{fig1.ps}
\end{figure}


\end{document}                    % DO NOT DELETE THIS LINE
%%%%%%%%%%%%%%%%%%%%%%%%%%%%%%%%%%%%%%%%%%%%%%%%%%%%%%%%%%%%%%%%%%%%%%%%%%%%%%
