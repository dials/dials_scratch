\documentclass{article}
\begin{document}
\noindent
Definitions:
\begin{itemize}
\item{$t_0$ sensor thickness}
\item{$\theta$ angle between incoming ray and detector normal}
\item{$\mu$ attenuation coefficient, i.e. probability of $e^{-\mu x}$ of photon reaching depth $x$ in material}
\item{$t$ apparent thickness equal to $\frac{t_0}{\cos \theta}$}
\end{itemize}

\noindent
The DQE may be calculated as

\begin{equation}
K \int_0^t e^{-\mu x} dx = K \left[ - \frac{1}{\mu} e^{-\mu x} \right]_0^t = 
K \frac{1}{\mu} \left( 1 - e^{-\mu t} \right)
\end{equation}

\noindent
modulo some constant term e.g. integral to infinity of same, divided
through to normalize - this will be $K = \mu$.
So for determination of the ``average'' shift in the central impact i.e. the shift for the centre of the energy deposition range simply need to solve 

\begin{equation}
\frac{1}{2} = \frac{\int_0^s e^{-\mu x} dx}{\int_0^t e^{-\mu x} dx}
\end{equation}

\noindent
for $s$, from which we may derive the offset $o = s \sin
\theta$. N.B. in all of these calculations no units have been
presented or are required, all that is necessary is that the units are
\emph{consistent}. Also N.B. the constant term referred to above
should drop out here anyhow...

Solving this equation gives (after a little math, on page 73 of my notebook)

\begin{equation}
s = - \frac{1}{\mu} \ln \left( \frac{1}{2} \left( 1 + e^{-\mu t} \right) \right)
\end{equation}

\noindent
which expanded in full (i.e. with the $\theta$ terms) gives

\begin{equation}
o = - \frac{\sin \theta}{\mu} \ln \left( \frac{1}{2} \left( 1 + e^{-\mu t_0 / \cos \theta} \right) \right)
\end{equation}

\noindent
where $o$ is the shift in the direction of the vector component of the incoming ray normal to the detector normal.

Aha on reflection this protocol is probably not correct - solving for
$s$ does not really determine where the energy is distributed as the
profile for this is asymmetric. In fact probably better to look at

\begin{equation}
o = \frac{\int_0^t \sin \theta x e^{-\mu x} dx}{\int_0^t e^{-\mu x} dx}
\end{equation}

\noindent
which would better express the ``average'' position on the detector
that the energy was deposited at, w.r.t. the incoming ray
position. Working this through gives

\begin{equation}
o = \frac{\left[ - \frac{1}{\mu^2} \sin \theta \left( \mu x + 1
    \right)e^{-\mu x} \right]_0^t}
{\left[ - \frac{1}{\mu} e^{-\mu x} \right]_0^t}
\end{equation}

\noindent
which I evaluate to

\begin{equation}
o = \frac{\sin \theta \left( \left( 1 + \mu \right) - \left( 1 + \mu t
    \right) e^{-\mu t}\right)} {\mu \left( 1 - e^{-\mu t} \right)}
\end{equation}

\noindent
where of course $t = \frac{t_0}{\cos \theta}$. When input this gave
more realistic extremes of offset in the range 
-0.8 to 0.8 pixels or so... 

\end{document}



