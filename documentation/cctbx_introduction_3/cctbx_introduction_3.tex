\documentclass[a4paper, 11pt]{article}
\usepackage{graphicx}
\title{CCTBX: Making Compiled Extensions}
\author{Graeme Winter, Diamond Light Source}

\begin{document}

\maketitle

\section{Introduction}

Much of the benefit of the hybrid approach to CCTBX\footnote{http://cctbx.sourceforge.net/current/tour.html\#thinking-hybrid} is the ability to move computationally expensive steps from Python to C++ relatively straightforwardly. It is however helpful to have an example of how to create C++ extensions and compile these with the existing compilation tools. The aim of this document is to provide this example with a toy example, toytbx. Some introductory information is however useful.

Within CCTBX C++ code is made available to Python using the Boost Python framework\footnote{http://www.boost.org/libs/python/doc/} and compiled with SCons\footnote{http://www.scons.org/}. Two steps are therefore needed to expose C++ code: writing a wrapper using Boost Python and a SConscript for compilation. Using this libtbx does most of the work. In addition it is nice to expose functionality as ``pseudo-executable'' programs e.g. toytbx.task - this is also handled automatically by libtbx.

\section{C++ Code}

The following code defines a trivial module which contains a single function, which itself generates a Python list. The code defines a module \verb|toytbx_ext|:

{\small
\begin{verbatim}
#include <boost/python.hpp>
#include <cctype>

namespace toytbx { 
  namespace ext {

    static boost::python::list make_list(size_t n)
    {
      boost::python::list result;
      for(size_t i = 0; i < n; i++) {
        result.append(i);
      }
      return result;
    }

    void init_module()
    {
      using namespace boost::python;
      def("make_list", make_list, (arg("size")));
    }

  }
} // namespace toytbx::ext

BOOST_PYTHON_MODULE(toytbx_ext)
{
  toytbx::ext::init_module();
}
\end{verbatim}
}

For more functions, extra static methods will be needed, as well as additional \verb|def| statements in the \verb|init_module| method.

\section{Compilation}

To compile the code above, SCons is used with libtbx. The first step is to define a SConscript which states the modules requirements:

{\small
\begin{verbatim}
import libtbx.load_env
Import("env_etc")

env_etc.toytbx_include = libtbx.env.dist_path("toytbx")

if (not env_etc.no_boost_python and hasattr(env_etc, "boost_adaptbx_include")):
  Import("env_no_includes_boost_python_ext")
  env = env_no_includes_boost_python_ext.Clone()
  env_etc.enable_more_warnings(env=env)
  env_etc.include_registry.append(
    env=env,
    paths=[
      env_etc.libtbx_include,
      env_etc.boost_adaptbx_include,
      env_etc.boost_include,
      env_etc.python_include,
      env_etc.toytbx_include])
  env.SharedLibrary(
    target="#lib/toytbx_ext",
    source=["ext.cpp"])
\end{verbatim}
}

This can refer to other compiled modules to ensure dependencies are correctly built. With this file and ext.cpp in the toytbx directory, accessible to libtbx.configure, libtbx.configure toytbx may be run, followed by make, which will correctly compile the extension module.

\section{Tidying}

This has generated an extension module, however it may be desirable to (i) import this nicely named as e.g. toytbx and (ii) add Python code to the module. This is best achieved by adding \verb|__init__.py| to the toytbx directory containing:

{\small
\begin{verbatim}
from __future__ import division
try:
  import boost.python
except Exception:
  ext = None
else:
  ext = boost.python.import_ext("toytbx_ext", optional = False)

if not ext is None:
  from toytbx_ext import *
\end{verbatim}
}

As with other code test cases should be written which ensure that the behaviour is correct. In this case, the following test is reasonable:

{\small
\begin{verbatim}
from toytbx import make_list

def tst_toytbx():
    assert(make_list(4) == [j for j in range(4)])
    print 'OK'

if __name__ == '__main__':
    tst_toytbx()
\end{verbatim}
}

After which \verb|cctbx.python tst_toytbx.py| should print 'OK'.

\section{Command Lines}

If a command-line executable is desired once again libtbx comes to the rescue. Make a directory within the toytbx called \verb|command_line|, and within there \verb|make_list.py|:

{\small
\begin{verbatim}
from toytbx import make_list

def main(args):
    assert(args)
    n = int(args[0])
    print make_list(n)

if __name__ == '__main__':
    import sys
    main(sys.argv[1:])
\end{verbatim}
}

Rerunning libtbx.configure and make should generate executables allowing:

{\small
\begin{verbatim}
Graemes-MacBook-Pro:toytbx graeme$ toytbx.make_list 10
[0, 1, 2, 3, 4, 5, 6, 7, 8, 9]
\end{verbatim}
}

\section{Acknowledgements}

This was prepared based on the fable module, after a pointer from Nat Echols and remembering things from Nick Sauter and Ralf Grosse-Kunstleve.

\end{document}